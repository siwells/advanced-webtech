\chapter{Build a Quiz Site}
\label{quiz}
\paragraph{} In this chapter we'll explore a few different ways to build a quiz site using various features of Flask to iteratively elaborate our site.


\section{A Super Simple Quiz}
\paragraph{} We'll start with a super simple example. On each page you'll be asked a question, and you'll be presented with a selection of possible answers. Each answer is a link to another page. If you click the link which is the correct answer then you'll move to the next question and if you click an incorrect answer link then you'll move to a page telling you that your answer was incorrect and offering an opportunity for another go.

\paragraph{} This demonstrates how we can decompose the problem ``building a quiz'' into a simple hypertext site, needing neither javascript in the client nor additional python on the server, to provide a simple quiz game experience. We have, however, included a little bit of inline HTML for each page in order to get some visual interest and to enable us to create hyperlinks.

\paragraph{} Obviously there are many ways to create a web-based quiz site, and this is just one super simple approach that shows off some aspects of Flask. So let's take a look.

\begin{lstlisting}
from flask import Flask
app = Flask(__name__)

@app.route("/")
def hello():
    return """
    <h1>A Super Simple Quiz Example</h1>

    <p>Hi folks.</p>
    <p>Welcome to this super simple quiz site example. It's meant to be a really straightforward proof of concept that demonstrates the simplest kind of online quiz using just some basic Python & Pure Flask (i.e. no static files or templates. Minimal HTML, CSS, & JS).</p>

    <b>Do you want to play a game?</b>

    <a href="/q1/">Damn right I want to play a game!</a>
    """

@app.route("/q1/")
def q1():
    return """
    <h1>Question One</h1>
    <p>Which is the best university in Edinburgh?</p>

    <ul>
        <li><a href="/q2/">Edinburgh Napier</a></li>
        <li><a href="/q1w/">University of Edinburgh</a></li>
        <li><a href="/q1w/">Herriot Watt</a></li>
        <li><a href="/q1w/">Queen Margaret</a></li>
    <ul> 
    """

@app.route("/q1w/")
def q1w():
    return """
    <h1>Das ist der wrong answer!</h1>

    <a href="/q1/">Do you want to try again?</a>
    """

@app.route("/q2/")
def q2():
    return """
    <h1>Question Two</h1>
    <p>Which is the best university in Scotland?</p>

    <ul>
        <li><a href="/q3/">Edinburgh Napier</a></li>
        <li><a href="/q2w/">University of Edinburgh</a></li>
        <li><a href="/q2w/">Herriot Watt</a></li>
        <li><a href="/q2w/">Queen Margaret</a></li>
    <ul> 
    """

@app.route("/q2w/")
def q2w():
    return """
    <h1>That answer was a little lacking in correctness.</h1>

    <a href="/q2/">Do you want to try again?</a>
    """


@app.route("/q3/")
def q3():
    return """
    <h1>Question Three</h1>
    <p>Which is the best university in the UK?</p>

    <ul>
        <li><a href="/success/">Edinburgh Napier</a></li>
        <li><a href="/q3w/">University of Edinburgh</a></li>
        <li><a href="/q3w/">Herriot Watt</a></li>
        <li><a href="/q3w/">Queen Margaret</a></li>
    <ul>
    """

@app.route("/q3w/")
def q3w():
    return """
    <h1>Afraid not!</h1>
    <p>Perhaps consider whether there was a pattern forming amongst the answers to previous questions?</p>

    <a href="/q3/">Do you want to try again?</a>
    """

@app.route("/success/")
def success():
    return """
    <h1>You answered all the questions correctly</h1>
    <h2>Well done you</h2>

    <a href="/">Let's return to the home page</a>

    """

if __name__ == "__main__":
    app.run(host="0.0.0.0")
\end{lstlisting}


\paragraph{} This example demonstrates the following:

\begin{itemize}
\item How to create a number of routes. Remember that each route corresponds to a different web address on your site. For simple examples, each web address also corresponds to a different web page (and to a different Python function). You should be noticing how the route, function, and page are related and work together.
\item A simple way of breaking an idea, a quiz site, into a number of pages that are inter-connected using hyperlinks.
\end{itemize}


\section{A Simple Quiz Using HTML Files}
\paragraph{} This version of the quiz is almost identical to the previous one, however, to make things easier, we've moved all of the HTML content for the pages out of the Python script, and into their own individual HTML files, one for each page. These HTML files are stored in their own sub-folder named `templates'. The templates folder is the default location\footnote{although this can be overridden} in which Flask looks for HTML templates. Flask will serve up any matching HTML file in the template folder when the render\_template() function is called. Note that a template is an HTML file that possibly contains additional JINJA2 code directives to dynamically alter the HTML content. However, we don't have any Jinja2 directives in our HTML files at this point, so our templates are the most straightforward that we can create, raw and valid HTML, which will be returned unaltered to our calling client.

\paragraph{} The project is organised as follows. We have a project folder containing our quiz.py script and a sub-folder called ``templates''. In the templates folder are eight HTML files that represent the pages of our quiz site.

The `quiz.py' file listing is as follows:

\begin{lstlisting}
from flask import Flask, render_template
app = Flask(__name__)

@app.route("/")
def hello():
    return render_template('index.html')

@app.route("/q1/")
def q1():
    return render_template('q1.html')

@app.route("/q1w/")
def q1w():
    return render_template('q1w.html')

@app.route("/q2/")
def q2():
    return render_template('q2.html')

@app.route("/q2w/")
def q2w():
    return render_template('q2w.html')

@app.route("/q3/")
def q3():
    return render_template('q3.html')

@app.route("/q3w/")
def q3w():
    return render_template('q3w.html')

@app.route("/success/")
def success():
    return render_template('success.html')
    
if __name__ == "__main__":
    app.run(host="0.0.0.0")
\end{lstlisting}

\paragraph{} This is a simple Flask app, very similar to the previous Quiz example, but we've extracted the HTML from within the Python file. The HTML is now stored in their own separate HTML files, one for each page. To send a specific HTML file back to the caller we use the render\_template() function passing in the name of the template to send. Note that each of our routes corresponds to a different page of the site and that each route returns a different template, hence, a different page.

\paragraph{} The `/' route is the index page, then we've named our other routes and associated functions to help keep track of their functionality, e.g. `/q1/' is question 1, and `/q1w/' is the page for incorrect answers to question 1. The same pattern is used for questions 2 and 3. Finally their is a success page that is reachable from the `/success/' route.

\paragraph{} Let's take a look at the contents of the `index.html' page:

\begin{lstlisting}
<!DOCTYPE html>
<html>
<head></head>
<body>
    <h1>A Super Simple Quiz Example</h1>

    <p>Hi folks.</p>
    <p>This time, instead of returning our HTML code as strings from our Python routes/functions, we've extracted the HTML code, topped and tailed it into complete HTML files, and placed them in the templates folder.</p>

    <b>Do you want to play a game?</b>

    <a href="/q1/">Damn right I want to play a game!</a>
</body>
</html>
\end{lstlisting}

\paragraph{} Note how the index page links directly to the first question page `q1.html', which is shown here:

\begin{lstlisting}
<!DOCTYPE html>
<html>
<head></head>
<body>
    <h1>Question One</h1>
    <p>Which is the best university in Edinburgh?</p>

    <ul>
        <li><a href="/q2/">Edinburgh Napier</a></li>
        <li><a href="/q1w/">University of Edinburgh</a></li>
        <li><a href="/q1w/">Herriot Watt</a></li>
        <li><a href="/q1w/">Queen Margaret</a></li>
    <ul>
</body>
</html>
\end{lstlisting}

\paragraph{} Note how each anser leads to a different page, the question 2 page for the correct answer and the question 1 wrong page for all other answers. The `q1w.html' page is shown here:


\begin{lstlisting}
<!DOCTYPE html>
<html>
<head></head>
<body>
    <h1>Das ist der wrong answer!</h1>

    <a href="/q1/">Do you want to try again?</a>
</body>
</html>
\end{lstlisting}

\paragraph{} Note how this page displays a message informing the player that they are wrong, and offers a link back to try answering the question again.

\paragraph{} The remaining HTML pages are similarly structured so it is worth investigating them in the Git repository and making sure you have a good understanding of how each works, both in terms of internal structure of a single page, but also how each page, through hyperlinks, connect to other specific pages that make up the site. In this way the hyperlinks between pages make up an aspect of the site's functionality.



\section{A Simple Quiz Using Static Files}
\paragraph{} This example expands upon the previous one and still includes both the HTML files in the templates folder and the core quiz.py script. However we've also added a static sub-folder as a sibling to the templates folder. Into this, for organisation purposes we have a sub-folder called `css', to store our CSS files which we'll beed to add some style to our HTML. CSS files are static because, at least in our current context, they won't change as a result of our Flask app running, and will generally remain the same whilst the site is being used. Javascript files are also considered static, as are image files, font files, and anything else that isn't being generated or altered by our Python code\footnote{Note that you could have CSS, JS, images, etc generated on the fly directly from code but under normal circumstances theses are generally considered to be fairly static parts of a web-site.}.

\paragraph{} Our quiz.py hasn't altered since the last example, our only changes this time around are within the HTML files, to indicate where the new CSS style files are located. So let's first look at the CSS file and then we'll look at a representative example of an HTML template file using that style file.

\paragraph{} In the static folder you will see a css folder. Within that folder you will find style.css which has the following contents:

\begin{lstlisting}
body{
    max-width:650px;
    margin:40px auto;
    padding:0 10px;
    font:18px/1.5 -apple-system,BlinkMacSystemFont,"avenir next",avenir,"Segoe UI","lucida grande","helvetica neue",helvetica,"Fira Sans",roboto,noto,"Droid Sans",cantarell,oxygen,ubuntu,"franklin gothic medium","century gothic","Liberation Sans",sans-serif;
    color:#444
    }
h1,h2,h3{line-height:1.2}

\end{lstlisting}

\paragraph{} This is fairly standard CSS to make our page look a little better than raw HTML. Obviously you can include whicheve CSS directives you want but this is meant to be a simple demonstration that improves on the default HTML style. We've set a max-width for the body, added a margin and some padding, specified some fonts, and a background colour. We've also set the line heigth for the level 1, 2, and 3 headings.

\paragraph{} How do we get our HTML templates to use this style? Let's look at a representative example, the `index.html' file:

\begin{lstlisting}[]
<!DOCTYPE html>
<html>
<head>
    <link href="{{ url_for('static', filename='css/style.css') }}" rel="stylesheet" />
</head>
<body>
    <h1>A Super Simple Quiz Example</h1>

    <p>Hi folks.</p>
    <p>This time, instead of returning our HTML code as strings from our Python routes/functions, we've extracted the HTML code, topped and tailed it into complete HTML files, and placed them in the templates folder.</p>

    <b>Do you want to play a game?</b>

    <a href="/q1/">Damn right I want to play a game!</a>


</body>
</html>
\end{lstlisting}

\paragraph{} We've added a link element into the head element of the page. Notice that this isn't just a regular HTML element though. This also includes a Jinja2 directive \emph{\{\{ url\_for('static', filename='css/style.css') \}\}}. Jinja2 uses double opening and closing curly braces ``\{\{'' and ``\}\}'' to delimit the start and end of Jinja2 code, which is interspersed within the HTML. In this case Jinja2 is using the url\_for() function to construct a hyperlink to place into the HTML at this point. It is also specifying that a specific file be referenced by this link, style.css, and that the file is located in the static folder. ALso notice that because style.css is in it's own `css' sub-folder within the static folder, we've just added the extra path information in the \emph{filename=} clause.

\paragraph{} If you example the other HTML files in the templates folder you'll see that a similar Jinja2 directive is used. This means that every HTML file is linking to the same CSS style file. The same mechanism can be used to reference any files that you place in your static folders, so if you want to include an image, e.g. a png or jpg file, or a javascript file, then you can use the url\_for() function in a similar way to place the correct path within the correct HTML hyperlink.


\section{A slighly more complex Quiz}
\paragraph{} This example is largely similar to the previous examples but makes some significant changes. Firstly, the static folders and CSS remain the same. However one might ask, why do we need an HTML file for each question if each page is actually similar. Couldn't we just create a template to represent, e.g. a generic question page, and then dynamically fill out the template with details when the page is called? Yes we can. It turns out that this is at the heart of the idea of dynamically and efficiently creating web pages using Flask.

\paragraph{} Firstly, example the templates folder. Notice that there are just four templates in our folder, half as many as previously. We still have an index.html and a success.html page, but now we have a generic quiz.html template used for each question page in the quiz and a generic `wrong answer' page, used when a player makes a poor choice of answer. Let's briefly look at the quiz and wrong answer pages in turn:


\begin{lstlisting}
<!DOCTYPE html>
<html>
<head>
    <link href="{{ url_for('static', filename='css/style.css') }}" rel="stylesheet" />
</head>
<body>
    <h1>Question {{ number }}</h1>
    <p> {{ text }} </p>

    <ul>
        
        <li><a href="/quiz/?answer={{loop.index}}">{{ answer }}</a></li> 
        
    <ul>
</body>
</html>

\end{lstlisting}

\paragraph{} Notice that in this template we now have three additional Jinja2 directives in addition to the one for the stylesheet link. To find the Jinja2 directives just look for the double curly braces. The first directive is in the $<$h1$>$> element and gives a placeholder for a question number to be inserted. The next directive is in the paragraph element and is a placeholder for question text. The final directive is a little more complex and uses a for loop to iterative over a list of answers, adding each as a list item in an HTML unordered list. This one is also more complex because it is made up of a few additional Jinja2 directives to specify a number in each $<$a$>$ tag and to create a placeholder for the answer text.

\paragraph{} Now let's take a look at the wrong answer template in the wrong.html file:

\begin{lstlisting}
<!DOCTYPE html>
<html>
<head>
    <link href="{{ url_for('static', filename='css/style.css') }}" rel="stylesheet" />
</head>
<body>
    <p> {{ text }} </p> 

    <a href="/">Let's return to the home page</a>
</body>
</html>

\end{lstlisting}

\paragraph{} This just has a single additional Jinja2 directive to allow some text to be placed within a paragraph element.

\paragraph{} In both templates, the data to use with the Jinja2 directive, e.g. the text to place in the paragraph, is supplied by the calling function in our Python Flask script.

\paragraph{} Now let's look at the quiz.py script. There've been some fairly major alterations here to enable us to make better use or our templates, and also, more importantly, we're now checking whether the answer is correct on the server and generating the response depending upon whether the player was right or wrong.

\begin{lstlisting}
from flask import Flask, render_template, request, session
app = Flask(__name__)
app.secret_key = 'SUPERSEKRETKEY'


@app.route("/")
def hello():
    session['question'] = 1
    return render_template('index.html')


@app.route("/quiz/")
def quiz():
    q = None
    qa = {
        "1":{
            "text":"Which is the best university in Edinburgh?",
            "answer":1,
            "answers":["Edinburgh Napier", "University of Edinburgh", "Heriott Watt", "Queen Mary" ]
        },
        "2":{
            "text":"Which is the best university in Scotland?",
            "answer":1,
            "answers":["Edinburgh Napier", "University of Edinburgh", "Heriott Watt", "Queen Mary" ]
        },
        "3":{
            "text":"Which is the best university in the UK?",
            "answer":1,
            "answers":["Edinburgh Napier", "University of Edinburgh", "Heriott Watt", "Queen Mary" ]
        },
        "4":{
            "text":"Which is the best university in the World?",
            "answer":1,
            "answers":["Edinburgh Napier", "University of Edinburgh", "Heriott Watt", "Queen Mary" ]
        }
    }
    try:
        if (session['question']):
            q = int(session['question'])
    except KeyError:
        q = 1

    answer = request.args.get('answer', None)
    if answer is not None:
        correct = qa.get(str(q)).get('answer')
        if str(answer) == str(correct):
            q = q+1
            session['question'] = q
            if q > len(qa):
                return render_template('success.html')
            else:
                return render_template('quiz.html', text=qa[str(q)]["text"], answers=qa[str(q)]["answers"], number=q)
        else:
            return render_template('wrong.html', text="Das ist der wrong answer!!!")
    else:
        return render_template('quiz.html', text=qa[str(q)]["text"], answers=qa[str(q)]["answers"], number=q)


@app.route("/success/")
def success():
    return render_template('success.html')
    

if __name__ == "__main__":
    app.run(host="0.0.0.0")
\end{lstlisting}

\paragraph{} Note that the root and success routes have not been altered greatly. The root route sets the current question in the session object to `1' but there are no further changes. All of the substantive changes in this version are in the new `/quiz/' route which is the heart of our quiz game.

\paragraph{} Lines 15 to 36 are a dictionary containing the questions and answers for our quiz. The `text' key is the text of the question, `answer' is the number of the answer in our answer list, and `answers' is a list of possible answers. Lines 37 to 41 check for a valid session and retrieve the current question if it exists or else set the current question to 1 if it doesnt'. The quiz route uses a URL argument parameter to communicate the players answer number back to the server, so we retrieve this in line 43. At this point we branch depending whether there is an answer or not. If not then we display page for the current question (line 56). Otherwise we check the answer, if it is wrong we render the template for the wrong answer and return it (line 54). If the answer is correct and there are more questions then we render the template for the next question (line 52) and if we've run out of question then we render the template for the successful culmination of the quiz (line 50). Note in the render\_template() calls how we supply various additional bits of data? These are the data that the Jinja2 directives use to complete the template into a fully-formed HTML page. This extra data in the render\_template() function is our conduit from Python code, data, and variables into our HTML templates.



%\begin{lstlisting}
%\end{lstlisting}

